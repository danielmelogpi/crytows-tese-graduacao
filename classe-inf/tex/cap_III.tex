\chapter{Uso do padr�o OpenPGP para criptografia}
\label{cap:usoPGPCripto}

% - - - - - - - - - - - - - - - - - - - - - - - - - - - - - - - - - - -
 PGP � uma fam�lia de softwares da �rea de seguran�a desenvolvidos inicialmente por Philip R. Zimmermann \cite{creatorOfPGPSilentCircle} e liberada como um freeware em 1991 e atualmente � mantida pela  PGP Corp, adquirida em 2010 pela Symantec.Tendo como base esta experi�ncia foi desenvolvido o padr�o OpenPGP, que cont�m a mesma proposta de criptografia por meio de chaves assim�tricas, uma p�blica e outra privada, mas agora com uma especifica��o publicada na RFC 4880 - OpenPGP Message Format \cite{RFC4880}. A publica��o desta especifica��o permitiu o nascimento de implementa��es abertas. A mais conhecida para desktop � a GnuPG, ou simplesmente GPG, tanto que, por vezes, os termos  PGP e GPG s�o usados de forma intercambi�vel.

Esse formato de comunica��o estabelece o sigilo da mensagem e o n�o-rep�dio \cite{lehtonen2002pattern} - incapacidade de uma das partes de negar que assinou a mensagem se, de fato, o fez -  da mensagem, tudo isso mantendo as chaves privadas - o recurso que guarda o poder de assinar e, portanto, de identifica��o - em sigilo.

Essa tecnologia encontrou um forte caso de uso nas trocas de e-mail, impedindo que a intercepta��o das mensagens comprometesse seu sigilo e, que um terceiro pudesse se passar por um dos interlocutores de forma despercebida ou, ainda, que um dos interlocutores mais tarde negasse que ele assinou a mensagem.

Outro caso de uso bastante explorado � a assinatura de arquivos. Dado que uma assinatura precisa da senha do chaveiro do usu�rio somada � posse da chave privada ela pode ser usada com prop�sitos legais na assinatura de documentos digitais.

GPG est� dispon�vel para todos os grandes sistemas operacionais, de esta��es desktop at� celulares e v�rias bibliotecas permitem desenvolvimento sobre esta tecnologia.
