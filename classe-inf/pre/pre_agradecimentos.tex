\begin{agradecimentos}
Agrade�o aos meus pais pelo amor, carinho e cuidado que me foram dados durante todos os meus anos de vida. Seu apoio foi o que me permitiu enfrentar a gradua��o, ainda que longe de minhas origens. Tamb�m � minha irm�, Keila Melo. N�o poderia querer algu�m melhor com quem crescer e descobrir o mundo. Apesar de n�o conviver com nenhum deles nesas fase da vida, sua presen�a � intensa e viva nos meus dias.

Deixo aqui minha gratid�o � minha tia, Maria Antonia de Sousa, que me acolheu por v�rios anos como a um filho quando da minha chegada nesta cidade, e, com seu grande cora��o, viabilizou este curso superior.

Aos grandes amigos que para sempre quero comigo: Ana Let�cia Herculano, que � um presente na vida de quem a conhece e com quem tanto aprendo a cada palavra trocada; a Bruno Nogueira de Oliveira, que sabe trazer uma alegria e vitalidade que sem d�vida busco absorver; � minha querida J�ssica Millene, uma pessoa de cora��o t�o doce e receptivo que n�o posso descrever. Que eu possa ter por muitos anos o Tratorzinho, o Birl e a Neguinha, com quem seria um privil�gio envelhecer.


Agrade�o ao meu orientador no desenvolvimento deste trabalho, Professor Me. Marcelo Akira Inuzuka, que soube executar seu papel de direcionar os meus esfor�os e o desenvolvimento de ideias neste trabalho. Sem a sua incessante dedica��o como docente e mestre n�o atingiria os resultados deste trabalho. 
\end{agradecimentos}


